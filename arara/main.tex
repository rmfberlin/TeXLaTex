% !TEX TS-program = xelatex
% arara: xelatex: {
% arara: --> shell: yes,
% arara: --> synctex: yes,
% arara: --> directory: subdir,
% arara: --> files: [ 
% arara: --> dok-ext-01.tex, dok-ext-01.tex, dok-ext-01.tex,
% arara: --> dok-ext-02.tex, dok-ext-02.tex, dok-ext-02.tex, 
% arara: --> dok-ext-03.tex, dok-ext-03.tex, dok-ext-03.tex, 
% arara: --> ]
% arara: --> }
% arara: xelatex: { 
% arara: --> shell: yes,
% arara: --> synctex: yes,
% arara: --> files: [main.tex, main.tex, main.tex] }
%	Autor: RMF Berlin
%	Version: 0.1
%	Erstellt: 07.06.2021
%	letzte Änderung: 07.06.2021
%	TeX/LaTeX-engine: Übersetzung mit xelatex
% Codierung: UTF-8
% DTK 03/2021

\AtBeginDocument{%
	\newcommand{\version}{Version: 0.1 \\im Juni~2021}%
}	% Die Version wird auch auf dem Deckblatt ausgegeben

\documentclass[12pt, headings=small]{scrreprt}	
\usepackage{xltxtra}	
\usepackage{microtype}
\usepackage{libertine}	
\usepackage{polyglossia}
\setdefaultlanguage[spelling=new,	babelshorthands=true]{german}
\usepackage{pdfpages}
\usepackage{blindtext}


\title{Mit \emph{arara} externe LaTeX-/PDF-Dokumente erzeugen und in eine Hauptdatei einfügen}
\subtitle{Beispiel-Dateien -- Hauptdatei}
\author{}
\date{\today}


\begin{document}
\maketitle
\tableofcontents
  \chapter{Eins}
		\label{cha:Eins}
		\blindtext[2]

		\blindtext[2]
	% chapter Eins (end) 

	\chapter{Zwei}
		\label{cha:Zwei}
		\blindtext[2]
		
		\blindtext[2]
	% chapter Zwei (end)
	\chapter{Drei}
		\label{cha:Drei}
		\blindtext[2]
		
		\blindtext[2]
	% chapter Drei (end)

  % \include{kap_a}
	% \include{kap_b}
	% \include{kap_c}
	\includepdf[pages=-, link, addtotoc={1,chapter,1,Titel ext-dok-01,cha:ext-dok-01}]{subdir/dok-ext-01}
	\includepdf[pages=-, link, addtotoc={1,chapter,1,Titel ext-dok-02,cha:ext-dok-02}]{subdir/dok-ext-02}
	\includepdf[pages=-, link, addtotoc={1,chapter,1,Titel ext-dok-03,cha:ext-dok-03}]{subdir/dok-ext-03}
	% \include{kap_n}
\end{document}